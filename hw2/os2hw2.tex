\documentclass[IEEEtran,letterpaper,10pt,notitlepage,draftclsnofoot,onecolumn]{article}

\usepackage{nopageno}
\usepackage{alltt}
\usepackage{float}
\usepackage{color}
\usepackage{url}
\usepackage{balance}
\usepackage{enumitem}
\usepackage{pstricks, pst-node}
\usepackage{geometry}
\geometry{textheight=9.5in, textwidth=7in}
\newcommand{\cred}[1]{{\color{red}#1}}
\newcommand{\cblue}[1]{{\color{blue}#1}}
\usepackage{hyperref}
\usepackage{textcomp}
\usepackage{listings}
\usepackage{titling}
\usepackage{graphicx}
\usepackage{url}
\usepackage{setspace}

\definecolor{dkgreen}{rgb}{0,0.6,0}
\definecolor{gray}{rgb}{0.5,0.5,0.5}
\definecolor{mauve}{rgb}{0.58,0,0.82}
\lstset{frame=tb,
  language=python,
  aboveskip=3mm,
  belowskip=3mm,
  showstringspaces=false,
  columns=flexible,
  basicstyle={\small\ttfamily},
  numbers=none,
  numberstyle=\tiny\color{gray},
  keywordstyle=\color{blue},
  commentstyle=\color{dkgreen},
  stringstyle=\color{mauve},
  breaklines=true,
  breakatwhitespace=true,
  tabsize=3
}
\begin{document}
\section{Name and such}
Group 11-3, OS2, HW2

\begin{enumerate}
\item 
\textbf{What do you think the point of this assignment was?}
To make us muck around in the really gross parts of the kernel. 
Having us write an I/O scheduler forces us to play around
in the real underbelly of the OS. Also, you made us practice
looking up things about kernel memory structures, which turns
out to be doable with some proper grepping/google. I'm sure
we'll see lists again soon....

\item 
\textbf{How did you approach the problem?}
First, we looked up LOOK and CLOOK and picked LOOK because
it sounded easier than wrapping at 0/MAX. Then I thought
about how I would build the queue, as I figured it would
be the most important part. It was. My first idea for 
queue building was that I was going to resort it in
init, which was very wrong, but I also thought that
the command to launch the I/O was done there. Then I 
was less foolish and figured out that was what dispatch 
was for, and also that if we are doing insertion sort like
you mentioned in class then I should probably be sorting
while I'm inserting in add\_request. From there it was just
me figuring out how to access things in the current queue
so I could put the new thing in the right spot.
I also edited dispatch so that it'd know when to turn around
but didn't take much.

\item
\textbf{Testing details}
The short answer: Lots of testing prints. Thank god for
printk I guess. But I did I/O in the machine and made sure
the traversal order made sense given the input to the best
of my ability. It seems to, although I really wish there
was a common fflush for printk so I didn't have to sort 
output, that made it hard to read things.
\item
\textbf{What did we learn?}
Most major is how to do patch files, add system files,
and using basic kernel hoodoo especially lists.
Also, that kernel level stuff in Linux is much more
accessible than I thought it would be. I was expecting
assembly hell, it turns out to be linked lists and regular
looking C. I'm sure I just haven't seen the real gross
macro stuff yet though....
Maybe that's the next assignment.
\end{enumerate}
\end{document}
