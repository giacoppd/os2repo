\documentclass[IEEEtran,letterpaper,10pt,notitlepage,draftclsnofoot,onecolumn]{article}

\usepackage{nopageno}
\usepackage{alltt}
\usepackage{float}
\usepackage{color}
\usepackage{url}
\usepackage{balance}
\usepackage{enumitem}
\usepackage{pstricks, pst-node}
\usepackage{geometry}
\geometry{textheight=9.5in, textwidth=7in}
\newcommand{\cred}[1]{{\color{red}#1}}
\newcommand{\cblue}[1]{{\color{blue}#1}}
\usepackage{hyperref}
\usepackage{textcomp}
\usepackage{listings}
\usepackage{titling}
\usepackage{graphicx}
\usepackage{url}
\usepackage{setspace}

\definecolor{dkgreen}{rgb}{0,0.6,0}
\definecolor{gray}{rgb}{0.5,0.5,0.5}
\definecolor{mauve}{rgb}{0.58,0,0.82}
\lstset{frame=tb,
  language=python,
  aboveskip=3mm,
  belowskip=3mm,
  showstringspaces=false,
  columns=flexible,
  basicstyle={\small\ttfamily},
  numbers=none,
  numberstyle=\tiny\color{gray},
  keywordstyle=\color{blue},
  commentstyle=\color{dkgreen},
  stringstyle=\color{mauve},
  breaklines=true,
  breakatwhitespace=true,
  tabsize=3
}
\begin{document}
\textbf{Group 11-3, OS2, HW3}

\begin{enumerate}
\item 
\textbf{What do you think the main point of this assignment was?}
The point of this assignment was to learn how to implement a device driver
as a module for the Linux kernel. Also to learn how to use Linux commands 
for setting up a device driver performing some crypto in transferring data with
the help of crypto function calls. 

\item 
\textbf{How did you approach the problem?}
Initially we looked at the difference between the sbd and 
the sbull implementations and we decided to go with sbull.
Once we selected our method, we looked up the function calls
to see where they originated from. Then we tried to make changes
in the sbull\_transfer function since this where the data transferring
will happen and to apply the crypto calls so that encryption and 
decryption would occur to and from the device. The encryption
applied was the Advanced Encryption Standard or AES.

\item
\textbf{Testing details}
There were some uses of printk to see what was going on at
certain points of the program. To test the encryption, there was a
key and key length involved that had to be passed in as a module
parameter. Once the key is passed in, then with the use of the dd
command, we can create an encrypted ramdisk dump and try to
write it back to the ramdisk for checking if the encryption occurred.

\item
\textbf{What did we learn?}
We learned how to compile and make the device driver kernel object.
Also learned how to SCP the device driver kernel object to the VM and
inserting the module. Learned how to read parts of the crypto library of
Linux.

\item
\textbf{gitlog}
\begin{itemize}
\item 5c7c802 was zhaobe, 16 hours ago, message: Fixing compile errors.
\item f1357fd was 1ryan3, 17 hours ago, message: This one makes!
\item ab8bbad was zhaobe, 19 hours ago, message: Made changes to Ryans initial sbull.
\item 3903944 was 1ryan3, 19 hours ago, message: Adding the rest of the sample files from ldd3
\item 10e824b was dominic, 23 hours ago, message: Added some flags
\item 1f7f326 was dominic, 23 hours ago, message: Added some flags
\item d9462bf was dominic, 23 hours ago, message: Moved it to the working directory
\item ff846e0 was dominic, 24 hours ago, message: Added a line for command line key entry and also removed the config.h bit
\item 57dc289 was dominic, 27 hours ago, message: Merge branch 'master' of https://github.com/giacoppd/os2repo
\item bce399c was dominic, 27 hours ago, message: git is making me but I'll merge over it
\item d177cad was 1ryan3, 28 hours ago, message: Finished(?) encryption
\item 03e06b2 was zhaobe, 3 days ago, message: Adding Makefile and cleaning up warnings.
\item 27f968b was 1ryan3, 3 days ago, message: Seems to work correctly. Threads ID themselves etc. 
\end{itemize}

\item
\textbf{Work Log}
Initial weekend: We all looked into and tried to do our own
research for the sbd or sbull implementations.
Expo week May 15: Ryan worked on the concurrency 3 assignment
and Benny worked on the compilation errors and testing.
Weekend of May 20: Ryan added sbull implementation. Benny checked
over the implementation and tried fixing it. Dom ran the compilation and
looked into the inserting modules and mounting process for the device.
Ryan tried to find out the issue of compatibility between deprecated things
in the Linux core or just compiling it on os class.
Monday May 22: Finishing up final edits, compilation, and making things work.

\end{enumerate}
\end{document}
