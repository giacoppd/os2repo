\documentclass[IEEEtran,letterpaper,10pt,notitlepage,draftclsnofoot,onecolumn]{article}

\usepackage{nopageno}
\usepackage{alltt}
\usepackage{float}
\usepackage{color}
\usepackage{url}
\usepackage{balance}
\usepackage{enumitem}
\usepackage{pstricks, pst-node}
\usepackage{geometry}
\geometry{textheight=9.5in, textwidth=7in}
\newcommand{\cred}[1]{{\color{red}#1}}
\newcommand{\cblue}[1]{{\color{blue}#1}}
\usepackage{hyperref}
\usepackage{textcomp}
\usepackage{listings}
\usepackage{titling}
\usepackage{graphicx}
\usepackage{url}
\usepackage{setspace}

\definecolor{dkgreen}{rgb}{0,0.6,0}
\definecolor{gray}{rgb}{0.5,0.5,0.5}
\definecolor{mauve}{rgb}{0.58,0,0.82}
\lstset{frame=tb,
  language=python,
  aboveskip=3mm,
  belowskip=3mm,
  showstringspaces=false,
  columns=flexible,
  basicstyle={\small\ttfamily},
  numbers=none,
  numberstyle=\tiny\color{gray},
  keywordstyle=\color{blue},
  commentstyle=\color{dkgreen},
  stringstyle=\color{mauve},
  breaklines=true,
  breakatwhitespace=true,
  tabsize=3
}
\begin{document}
\section{Names} 
Dominic Giacoppe, Benny Zhao, Ryan Wood, Group 11-03
\section{A log of commands used to perform the requested actions.}
\begin{enumerate}
\item Open two terminals and in the first terminal:
\item cd /scratch/spring2017/11-03
\item git clone git://git.yoctoproject.org/linux-yocto-3.14
\item git checkout v3.14.26
\item cp /scratch/spring2017/files/config-3.14.26-yocto-qemu .config
\item make menuconfig to get a new window
\item inside the window: press / and type in LOCALVERSION, hit enter
\item press 1, press enter and edit the value to -11-03-hw1
\item make -j4 all
\item cd .. 
\item gdb
\item In the second terminal:
\item source /scratch/opt/environment-setup-i586-poky-linux
\item cd /scratch/spring2017/11-03
\sloppy
\item qemu-system-i386 -gdb tcp::5613 -S -nographic -kernel linux-yocto-3.14/arch/x86/boot/bzImage  -drive file=core-image-lsb-sdk-qemux86.ext3,if=virtio-enable-kvm -net none -usb -localtime --no-reboot --append "root=/dev/vda rw console=ttyS0 debug"
\item if you are getting file not found errors something got deleted when we weren't looking
\item else it should just hang here
\item Back to the first terminal
\item target remote :5613
\item continue
\item root, press Enter
\item you're in, do what you want in your linux environment.
\end{enumerate} 

\section{Explanation of each flag in the listed qemu command line}
\begin{enumerate}
\item -gdb :Waits for the gdb connection to device 
\item tcp::5613 :Opens connection to port 5613
\item -S :Do not start CPU at startup (requires user to type 
  "c/continue" in the monitor)
\item -nographic :Disables graphical output so QEMU is a 
  simple command line application
\item -kernel <bzImage> :Use the specified bzImage as kernel image 
  and kernel can be either a Linux kernel or multiboot format
\item -drive file=<file>, if=<interface> :Define a new drive where file= 
  is the disk image to use and if= defines which type of interface the drive is connected
\item -enable-kvm :Enables KVM full virtualization support
\item -net :Indicates that no network devices should be configured
\item -usb :Enables the USB driver
\item -localtime :Required for the correct date in MS-DOS or Windows
\item --no-reboot :Exit instead of rebooting
\item --append :Use cmdline as the kernel command line
\end{enumerate}

\section{Answer the following questions in sufficient detail (for the concurrency)}
\begin{enumerate}
\item \textbf{What do you think the main point of this assignment is?}
\begin{itemize}
\item The main point of this assignment was to engage the student 
  with the concept of concurrency and how to think in parallel. By parallel, 
  we mean to learn how threads work and how to synchronize them with the use 
  of mutexes/locks and to prevent deadlock.
\end{itemize}

\item \textbf{How did you personally approach the problem? Design decisions, algorithm, etc.}
\begin{itemize}
\item To approach the problem, our team first added the data structures 
  we plan to use later in the functions and then we laid out the skeleton 
  code of how we want our code to run. The initial design was to pass in 
  arguments to create our producer and consumer threads. But after some 
  debugging with gdb, we came to a dead end in what we were trying to 
  achieve. Therefore, one of our group members went to get clarification 
  from the professor about what exactly are we achieving here in terms 
  of the programming I/O. We got back together as a group and discussed 
  what changes we wanted to make towards our code design. The most 
  recent design consisted of the producer function to continuously 
  produce and our consumer function to consume when a resource is available
  independent of user input.
\end{itemize}

\item \textbf{How did you ensure your solution was correct? Testing details, for instance.}
\begin{itemize}
\item To ensure our solution was correct, we used gdb to debug our 
  initial code by checking when the consumer is attempting to consume 
  an empty buffer and when the producer is trying to produce on a full 
  buffer. We also developed the random generator implementation file 
  independent of the main file, so that we could confirm there would 
  be no issues from the random generator. Then we refactored our code 
  according to the clarifications we were provided.
\end{itemize}

\item \textbf{What did you learn?}
\begin{itemize}
\item For this assignment, we learned how to utilize pthreads in more 
  depth and understanding how the locks fundamentally work in this problem context.
  Also, that globals do indeed have a place in the world, and to make sure to pester
  higher command when clarification on requirements is needed
\end{itemize}
\end{enumerate}

\section{Work Log}
\begin{itemize}
\item 4/10 Went to lab, met each other, spend our hour trying to compile the kernel. Dominic took charge to finish it
\item 4/12 Met after class, Dominic demonstrated how he compiled the kernel. Agreed to look over the concurrency homework separately over the weekend and come up with code.
\item 4/18 Met after class. Started working on the baseline of the code until lab, then worked some more after lab. Agreed to have Ryan get the assembly bit working, and Dominic and Benny get the rest of it working.
\item 4/19 Ryan gets the assembly working, Dominic creates the first draft that would turn out to be not what the assignment wanted. Benny gets it to compile and run, but we all agree to meet tomorrow in an attempt to fix it.
\item 4/20 Team meets in Kelley after class. We spend 2 hours debugging it before speaking more with Kevin to determine that our solution was not what he wanted really, and that what he wanted was much easier. Dominic then spends 30 minutes revising it and gets a mostly working copy going.
\item 4/21 Team meets to look over this document and check the code before submission. It goes well.
\end{itemize}
\section{Version Control History, newest to oldest}
\begin{itemize}
\item d4cac22 was dominic, 9 hours ago, message: Apparently this is how locks should work
\item a7a2c07 was dominic, 9 hours ago, message: I think the pointer is fine, and it's not the curitem pointer is it?
\item 77a2f50 was dominic, 9 hours ago, message: Last issue: why does assign fail?
\item 1a192cb was dominic, 10 hours ago, message: Made adjustments based on feedback from kevin about what this actually needed to do
\item 8731d6b was dominic, 10 hours ago, message: Moved HW into it's own folder
\item 82529f8 was 1ryan3, 11 hours ago, message: Fixed a type mismatch
\item 1990553 was Dominic Giacoppe, 26 hours ago, message: Update os21.c
\item af90d8f was Dominic Giacoppe, 26 hours ago, message: Fixed last errors
\item 628508b was zhaobe, 26 hours ago, message: One error left for os21.c
\item 458b23c was 1ryan3, 28 hours ago, message: Source for random \# generation
\item 21fc53e was Dominic Giacoppe, 2 days ago, message: Update os21.c
\item 084972a was Dominic Giacoppe, 2 days ago, message: Add files via upload
\end{itemize}
\end{document}
